\chapter{Optimización de pesos por evento y ventana temporal}
\label{AnexoOptuna}

Este anexo documenta los resultados del proceso de optimización bayesiana realizado con \texttt{Optuna}, cuyo objetivo fue estimar los pesos relativos de cada tipo de evento y de cada ventana temporal para la construcción del \textit{score} de preferencia cliente–producto.  
El procedimiento se orientó a maximizar la métrica \textit{Precision@10} sobre un conjunto de validación temporal, lo que garantizó un balance adecuado entre la relevancia de las señales recientes y la estabilidad de los patrones históricos.

\section{Estrategia de optimización}

La optimización se ejecuta mediante búsqueda bayesiana a lo largo de 100 iteraciones.  
En cada \textit{trial} se ajustan simultáneamente los pesos temporales y los pesos por tipo de evento ($\alpha_{e}$), aplicándolos en la generación del \textit{score} compuesto de afinidad.  
La métrica de desempeño se calcula mediante el uso de ventanas móviles de seis meses, con el entrenamiento basado en el histórico comprendido entre los meses $N-7$ y $N-2$, y la validación de la capacidad predictiva sobre el mes $N-1$.

Los resultados evidencian un patrón consistente: las señales más recientes (1M) aportan mayor información predictiva que las históricas (6M), y los eventos transaccionales tienden a tener un peso superior al de los digitales, especialmente aquellos asociados a recomendaciones personalizadas o flujos de recompra recurrente.

\section{Pesos óptimos por ventana temporal}

La tabla~\ref{tab:window_weights} presenta los valores óptimos obtenidos para cada horizonte temporal.  
Se observa una clara preferencia hacia las señales más recientes, lo que indica que los comportamientos de compra recientes aportan mayor valor predictivo que los históricos, en línea con la dinámica de rotación del portafolio en el entorno B2B.

\begin{table}[H]
\centering
\caption[Pesos óptimos por ventana temporal]{Pesos óptimos obtenidos para cada horizonte temporal.}
\begin{tabular}{l c}
\toprule
\textbf{Ventana temporal} & \textbf{Peso ($\beta_w$)} \\
\midrule
1 mes (reciente) & 0{,}3925 \\
3 meses (intermedia) & 0{,}4735 \\
6 meses (larga) & 0{,}1340 \\
\bottomrule
\end{tabular}
\label{tab:window_weights}
\end{table}

\section{Pesos óptimos por tipo de evento}

En la tabla~\ref{tab:event_weights} se muestran los pesos óptimos estimados para cada tipo de evento.  
Las señales vinculadas a compras efectivas y órdenes generadas (\texttt{BUYER}, \texttt{ordered\_*}) son las más relevantes, seguidas por aquellas relacionadas con interacciones promocionales o de exposición de producto.  
Este patrón refuerza la importancia de las señales transaccionales en la predicción de recompra y en la calibración del \textit{score} de afinidad.

\begin{table}[H]
\centering
\caption[Pesos óptimos por tipo de evento]{Pesos óptimos estimados para cada tipo de evento (\textit{event\_weights}).}
\begin{tabularx}{0.9\textwidth}{l >{\centering\arraybackslash}X}
\toprule
\textbf{Evento} & \textbf{Peso ($\alpha_e$)} \\
\midrule
\texttt{BUYER} & 0{,}2208 \\
\texttt{ordered\_QUICK\_ORDER} & 0{,}1170 \\
\texttt{ordered} & 0{,}0700 \\
\texttt{card\_viewed\_QUICK\_ORDER} & 0{,}1318 \\
\texttt{card\_viewed\_FORGOTTEN\_ITEMS} & 0{,}0630 \\
\texttt{details\_page\_viewed} & 0{,}0066 \\
\texttt{card\_viewed\_CROSS\_SELL\_UP\_SELL} & 0{,}0916 \\
\texttt{ordered\_CROSS\_SELL\_UP\_SELL} & 0{,}1882 \\
\texttt{ordered\_FORGOTTEN\_ITEMS} & 0{,}2174 \\
\texttt{ordered\_RECENT\_SEARCHES} & 0{,}0734 \\
\texttt{ordered\_CLUB\_B} & 0{,}1008 \\
\texttt{ordered\_POPULAR\_SEARCHES} & 0{,}1896 \\
\texttt{removed} & 0{,}1228 \\
\texttt{card\_viewed\_RECENT\_SEARCHES} & 0{,}1786 \\
\texttt{card\_viewed\_POPULAR\_SEARCHES} & 0{,}2028 \\
\texttt{card\_viewed} & 0{,}0132 \\
\texttt{card\_viewed\_CLUB\_B} & 0{,}0128 \\
\bottomrule
\end{tabularx}
\label{tab:event_weights}
\end{table}

\section{Análisis e interpretación}

Los pesos reflejan una jerarquía coherente con el proceso de compra en la plataforma BEES: las órdenes efectivas (\texttt{BUYER}, \texttt{ordered\_*}) constituyen las señales más predictivas de recompra, seguidas por las interacciones promocionales y de exposición de producto (\texttt{card\_viewed\_*}).  
Las categorías vinculadas a mecanismos de recomendación específicos, como \textit{Cross-Sell/Up-Sell} y \textit{Forgotten Items}, presentan una fuerte correlación con la conversión, lo que respalda su priorización en el cálculo del \textit{score} final.  

En contraste, los eventos de exploración (\texttt{details\_page\_viewed}) o de fricción (\texttt{removed}) tienen menor peso, lo que confirma que su valor informativo es limitado cuando se los considera de forma aislada.

El conjunto de ponderaciones resultante se utiliza en la construcción del \textit{score} de preferencia compuesto, definido en la ecuación~\ref{eq:preference_score}donde $\alpha_e$ representa el peso por tipo de evento y $\beta_w$ el peso temporal.

\begin{equation}
\label{eq:preference_score}
\text{Score}_{ij} = \sum_{e \in E} \sum_{w \in W} \alpha_e \cdot \beta_w \cdot x_{ij}^{(e,w)}
\end{equation}

Este valor resume la intensidad y actualidad de las interacciones del cliente $i$ con el producto $j$, y constituye el insumo principal de la matriz cliente–producto empleada en los modelos de recomendación.
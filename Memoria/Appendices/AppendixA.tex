\chapter{Optimización de pesos por evento y ventana temporal}
\label{AnexoOptuna}

Este anexo documenta los resultados del proceso de optimización bayesiana realizado con \texttt{Optuna}, cuyo objetivo fue estimar los pesos relativos de cada tipo de evento y de cada ventana temporal para la construcción del \textit{score} de preferencia cliente-producto.  
El procedimiento se orientó a maximizar la métrica \textit{Precision@10} sobre un conjunto de validación temporal, garantizando un balance adecuado entre la relevancia de las señales recientes y la estabilidad de los patrones históricos.

\section{Estrategia de optimización}

La optimización se ejecutó sobre 100 iteraciones mediante búsqueda bayesiana.  
En cada \textit{trial} se ajustaron simultáneamente los pesos temporales y los pesos por tipo de evento ($\alpha_{e}$), aplicándolos en la generación del \textit{score} compuesto de afinidad.  
La métrica de desempeño se calculó utilizando ventanas móviles de seis meses, entrenando con el histórico comprendido entre los meses $N-7$ y $N-2$, y validando la capacidad predictiva sobre el mes $N-1$.

El resultado final mostró un patrón consistente: las señales más recientes (1M) aportan mayor información predictiva que las históricas (6M), y los eventos transaccionales tienden a tener mayor peso que los digitales, especialmente aquellos asociados a recomendaciones personalizadas o a flujos de compra recurrentes.

\section{Pesos óptimos por ventana temporal}

\begin{table}[H]
\centering
\caption{Pesos óptimos de las ventanas temporales.}
\begin{tabular}{lc}
\toprule
\textbf{Ventana} & \textbf{Peso} \\
\midrule
1 mes (reciente) & 0.3925 \\
3 meses (intermedia) & 0.4735 \\
6 meses (larga) & 0.1340 \\
\bottomrule
\end{tabular}
\label{tab:window_weights}
\end{table}

Los pesos asignan mayor relevancia a los comportamientos más recientes, reflejando la mayor capacidad predictiva de las señales cercanas en el tiempo.

\section{Pesos óptimos por tipo de evento}

\begin{table}[H]
\centering
\caption{Pesos óptimos por tipo de evento (\textit{event\_weights}).}
\begin{tabular}{lc}
\toprule
\textbf{Evento} & \textbf{Peso ($\alpha_e$)} \\
\midrule
\texttt{BUYER} & 0.2208 \\
\texttt{ordered\_QUICK\_ORDER} & 0.1170 \\
\texttt{ordered} & 0.0700 \\
\texttt{card\_viewed\_QUICK\_ORDER} & 0.1318 \\
\texttt{card\_viewed\_FORGOTTEN\_ITEMS} & 0.0630 \\
\texttt{details\_page\_viewed} & 0.0066 \\
\texttt{card\_viewed\_CROSS\_SELL\_UP\_SELL} & 0.0916 \\
\texttt{ordered\_CROSS\_SELL\_UP\_SELL} & 0.1882 \\
\texttt{ordered\_FORGOTTEN\_ITEMS} & 0.2174 \\
\texttt{ordered\_RECENT\_SEARCHES} & 0.0734 \\
\texttt{ordered\_CLUB\_B} & 0.1008 \\
\texttt{ordered\_POPULAR\_SEARCHES} & 0.1896 \\
\texttt{removed} & 0.1228 \\
\texttt{card\_viewed\_RECENT\_SEARCHES} & 0.1786 \\
\texttt{card\_viewed\_POPULAR\_SEARCHES} & 0.2028 \\
\texttt{card\_viewed} & 0.0132 \\
\texttt{card\_viewed\_CLUB\_B} & 0.0128 \\
\bottomrule
\end{tabular}
\label{tab:event_weights}
\end{table}

\section{Análisis e interpretación}

Los pesos reflejan una jerarquía coherente con el proceso de compra en la plataforma BEES: las órdenes efectivas (\texttt{BUYER}, \texttt{ordered\_*}) constituyen las señales más predictivas de recompra, seguidas por las interacciones promocionales y de exposición de producto (\texttt{card\_viewed\_*}).  
Las categorías vinculadas a mecanismos de recomendación específicos, como \textit{Cross-Sell/Up-Sell} y \textit{Forgotten Items}, presentan una fuerte correlación con la conversión, lo que respalda su priorización en el cálculo del \textit{score} final.  

En contraste, los eventos de exploración (\texttt{details\_page\_viewed}) o de fricción (\texttt{removed}) tienen menor peso, lo que confirma que su valor informativo es limitado cuando se los considera de forma aislada.  

El conjunto de ponderaciones resultante fue utilizado para construir el \textit{score} de preferencia compuesto en la ecuación \ref{eq:preference_score}:

\begin{equation}
\label{eq:preference_score}
\text{Score}_{ij} = \sum_{e \in E} \sum_{w \in W} \alpha_e \cdot \beta_w \cdot x_{ij}^{(e,w)}
\end{equation}

donde $\alpha_e$ representa el peso por tipo de evento y $\beta_w$ el peso temporal.  
Este valor resume la intensidad y actualidad de las interacciones del cliente $i$ con el producto $j$, constituyendo el insumo principal de la matriz cliente-producto empleada en los modelos de recomendación.